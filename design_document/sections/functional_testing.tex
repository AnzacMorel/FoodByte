Order and menu operations were heavily tested with JUnit tests, this is because they provide the key operations for the application to fully function when processing customer orders, or adding menu items to the menu. The functions used for these operations had a high chance of being implemented improperly, such as modifying an order, or changing the price of menu items. Thus, it is vital to test the functions that support these operations as they will underpin our application upon release. JUnit tests were also written for model classes as they contain vital information, and provide a base for some of the functionalities of the application. GUI classes are not automatically tested at all as there is no viable or effective way to test these classes beyond manual testing. \\ \\
Some of the issues that we encountered while developing these tests were related to the heavy involvement of GUI testing, and not necessarily on the actual code itself. Another issue we encountered was waiting on implementation of features/codes before these tests could be created, as some tests required several areas of code to be functional. The last issue was the struggle of understanding the complexity of the system as a whole, the architecture of the system, and how each element worked together to make the system run. \\ \\ Issues that were encountered with the creation of unit testing were mainly related to understanding Cucumber and how it could be used to implement our acceptance testing. \\

