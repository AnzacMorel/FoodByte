Order and menu operations were heavily tested with JUnit tests, this is because they provide the main operations for the application to fully function such as processing customer orders, or adding menu items to the menu. The functions used for these operations had a big chance of not being implemented properly, such as if an order is being modified, or the price of the menu item is being changed. Thus, it is vital to test the functions that help with these operations as they will greatly impact our application upon release. JUnit tests were also written for model classes as they contain vital information, and the base for some of the functionaliies for this application. GUI classes is an area that were not tested at all, and this is maily because there is no viable way to test these classes. \\ \\
Some of the issues that we encountered while coming up with these tests were because some of them were more on the GUI testing, and not necessarily on the actual code itself. Another issue we encountered was because we had to wait for other features/codes to be implemented before creating these tests, as they were needed to make the testing work. Last issue was basically just trying to understand the complexity of the system as a whole, the architecture of the system, and how everything all work together to make the system run. \\ \\ Issues that were encountered with the creation of unit testing were mainly just trying to understand Cucumber and how to use that for creating our acceptance testing. \\

