Currently, FoodByte has the ability to import/export XML files so users can save the data each day if they wish or import new data into their system from another XML file with the same file structure. 
\\ \\
\noindent Users are able to add items via the management tab. The create Item popup will prompt the user to fill out the form which includes adding a name, cost price and selling price. Optionally, the user can also add a recipe and/or ingredients which may be other existing items. These items may also be edited by clicking the pen on the Item screen, next to the item which the user wants to edit. Stock can be added for these items via the Stock tab in the management screen which prompts the user via a popup for an expiry date if the expires checkbox is ticked along with quantity. The stock screen will only display stock if the date is still valid and the stock is not expired, for the stock to show it must also have a quantity greater than 0.
\\ \\
\noindent FoodByte also has the ability to create menu items prompting the user to choose an associated item from a drop-down box as well as enter a description and a selling price which overrides the selling price. This allows the user to change prices of menu items as they wish. These menu items can then be added to a menu where menus can be created from the Menu tab, menus can be edited as well by clicking the pen next to the associated menu.
\\ \\
\noindent Creating orders is simple, the user may add menu items to an order via the main order screen by clicking the menu item button from the grid of menu items. If the item is in stock, it will appear in the list of menu items currently in the order displayed on the right side of the order screen. Users can simply edit or remove these menu items by clicking the corresponding pen or bin icons accordingly. When orders are finalised, orders will be printed via the command prompt in the form of a receipt and a chefs order. The receipt includes the FoodByte name, date, order number, payment method, the price of each item as well as a list of order items, if the item is built from other ingredients, this will be represented via a tree structure. At the end of the receipt, it will display a total price of the order. The chefs order slip is a simplified version of this with the order number and tree structure of the order items in the order. Refunds may be completed on an order by selecting the refund button on the past order tab of the management screen.
\\ \\
\noindent FoodByte also manages the amount of cash in the register which increases when an order is paid for via cash or a refund is cancelled, and decreases when an order is refunded. 
