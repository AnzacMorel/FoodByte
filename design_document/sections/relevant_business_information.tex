FoodByte is worth developing because there is a gap in the software market for an application which is able to digitally manage day to day operations such as creating orders, managing stock as well as have a count of what is currently in the register all in one small package, and is designed for small businesses. FoodByte is able to provide the owners of the business with a tool to enhance and ease their order systems whilst providing a wide range of backend support for stock and item management.
\\ \\
The application will also allow for owners to import existing data files, such as xml files, containing data on products they offer into the system where the system will update all relevant sections in the application. Having the ability to import data allows the business load different sets of menus and items. The owner is also able to export the menu if they wish which will save the contents to an external file. Having the ability to export data allows the owner to keep track of how their business was doing at a particular time.
\\ \\
FoodByte is unique. One of the key selling points comes when adding new food items to the inventory, the high-level inheritance structure allow the user to have multiple variants of the same item. This allows for on the fly switching between variants of items, such as switching a burger to use gluten-free buns. This would greatly benefit some business as each item is independent of the next where customisations are concerned, for example; A customer may order a Cheese-burger but want to add more cheese, they can with the click of a few buttons.
\\ \\
The target market for the application is small food truck and cafe owners. Suitable customers could be a food truck stall at the Sunday market or a café at the University of Canterbury.