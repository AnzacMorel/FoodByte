\subsection{Andrew}

\subsection{Anzac}
Key lesson 1:
I learnt a lot about testing. Testing was not one of my strong suits. After seng201 I had not done a lot of testing and I was keen to remedy that with this project. After taking up the task of writing the acceptance testing as well as writing Junit tests for classes and performing manual tests etc. I have become comfortable with many different forms of testing and the risk that I took to do something that I was not familiar with definitely paid off.

Key lesson 2:
I learnt that teamwork is very important and working as a team is also very important. At the beginning of the course when the team were all working as individuals to complete tasks before the next meeting not a lot of work was being done and it was not to a high standard. However when we switched to more group based work with peer programming sessions the level and quantity of work increased. 

Key lesson 3:
I learnt that the design document is definitely not just something that you create before coding begins and is never looked at again after that. It is an essential part of any system and should be kept updated so that it accurately reflects its associated system  at any point in time. It is important as a point of reference for if you get stuck or need a new member joins the team etc. I also learnt that a lot of time needs to be put into this document and planning of a system before any coding can begin as they are both integral parts of the development process. 


\subsection{Connor}
The most important lesson I've learned from this project is the need for personal accountability for group work. This comes in many forms but having reliable means of communication and well organised meetings is very beneficial. As the project went on our test coverage got lower and I started to feel the effects and understand why test driven development is so so crucial. For more technical reflections see Section \ref{subsec:TechnicalDesignPost}.

\subsection{George}

\subsection{Hamesh}

\subsection{Rchi}

\subsection{Taran}
Key lesson 1:
The development of the GUI is not to be underestimated. Our original plan had only me assigned to developing the GUI as I believed that it would not be that hard as I had already done the JavaFX lab. This turned out to be doubly wrong as not only did I have to actually learn JavaFX, as the lab taught nothing but the fact that it existed and SceneBuilder basics, but the sheer quantity of work required for the GUI turned out to be collectively the same amount if not more than the back end architecture.
The lesson to be learnt from this is both that when learning a new tool it is better to assume you don't how to use it until proven otherwise and that the GUI will always be a very large part of any project.

Key lesson 2:
All the work that went into planning more than paid off. This can best be shown with the work that was put into planning the GUI. All of the sections that were well planned, mainly the operations side, were able to be rapidly developed and implemented to a high standard as there was already a template to follow. This can conversely be shown with details on the management side where there was little to no detailed planning. This caused several different minor variations of styles to occur, as well as struggles when considering where to put elements in the GUI as there was no pre thought towards leaving space for later elements.
The lesson that can be taken from this is that although planning can sometimes be a slog it is  more than worth the investment of time during the early stages.
