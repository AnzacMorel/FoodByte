Taran \\ \\
Key lesson 1: \\
The development of the GUI is not to be underestimated. Our original plan had only me assigned to developing the GUI as I believed that it would not be that hard as I had already done the JavaFX lab. This turned out to be doubly wrong as not only did I have to actually learn JavaFX, as the lab taught nothing but the fact that it existed and SceneBuilder basics, but the sheer quantity of work required for the GUI turned out to be collectively the same amount if not more than the back end architecture.
The lesson to be learnt from this is both that when learning a new tool it is better to assume you don't how to use it until proven otherwise and that the GUI will always be a very large part of any project.
\\ \\
Key lesson 2: \\
All the work that went into planning more than payed off. This can best be shown with the work that was put into planning the GUI. All of the sections that were well planned, mainly the operations side, were able to be rapidly developed and implemented to a high standard as there was already a template to follow. This can conversely be shown with details on the management side where there was little to no detailed planning. This caused several different minor variations of styles to occur, as well as struggles when considering where to put elements in the GUI as there was no pre thought towards leaving space for later elements.
The lesson that can be taken from this is that although planning can sometimes be a slog it is  more than worth the investment of time during the early stages.
\\ \\
\noindent Hamesh \\ \\
\noindent Key lesson 1: \\
\noindent Testing is a critical part when it comes to any sort of programming and implementing JUnit tests before implementing code to complete the tasks is much easier than trying to work it the other way around. I learned that manually testing the application is nowhere near as easy as just running the JUnit tests before pushing to git.
\\ \\
Key lesson 2: \\
Investing into a great program architecture at the start really pays off as it allows for easier modification to enable the use of larger feature sets. Implementing these complex features also expanded my knowledge in Java significantly having recursive methods within the application previously had seemed too scary, though this experience has expanded my confidence.
\\ \\
Key lesson 3: \\
More meetings and pair programming session dramatically improved productivity, also helps to get to know your team members better. 